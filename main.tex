\documentclass[sigconf,nonacm]{acmart}

%% Enable subfigures
\usepackage{subfigure}
%% Enable numbers in scientific format.
\usepackage{siunitx}
%% Enable enumerate start from.
\usepackage{enumitem}

%% Enable theorems
\newtheorem{theorem}{Theorem}[section]
\newtheorem{lemma}[theorem]{Lemma}

%% Enable algorithms
\usepackage{algorithm}
\usepackage[noend]{algpseudocode}
\let\ReturnInline\Return
\renewcommand{\Return}{\State\ReturnInline}
\algrenewcommand\algorithmicrequire{$\rhd$}
\algrenewcommand\algorithmicensure{$\square$}

%% Fonts used in the template cannot be substituted; margin 
%% adjustments are not allowed.
\AtBeginDocument{%
  \providecommand\BibTeX{{%
    \normalfont B\kern-0.5em{\scshape i\kern-0.25em b}\kern-0.8em\TeX}}}

%% Rights management information.
\setcopyright{acmcopyright}
\copyrightyear{2018}
\acmYear{2018}
\acmDOI{XXXXXXX.XXXXXXX}

%% These commands are for a PROCEEDINGS abstract or paper.
\acmConference[Conference acronym 'XX]{Make sure to enter the correct
  conference title from your rights confirmation emai}{June 03--05,
  2018}{Woodstock, NY}
%% Title of the proceedings is different from ``Proceedings of ...''?
% \acmBooktitle{Woodstock '18: ACM Symposium on Neural Gaze Detection,
%  June 03--05, 2018, Woodstock, NY} 
% \acmPrice{15.00}
% \acmISBN{978-1-4503-XXXX-X/18/06}

%% Submission ID.
% \acmSubmissionID{123-A56-BU3}

%% Use the "author year" style of citations and references?
% \citestyle{acmauthoryear}

%% Message
\newcommand{\kk}[1]{{{\color{red} #1}}}
\newcommand{\ds}[1]{{{\color{blue} #1}}}
\newcommand{\su}[1]{{{\color{green} #1}}}

%% Ignore block
\newcommand{\ignore}[1]{}

%% Macros
\newcommand{\Lou}{\textit{Louvain}}
\newcommand{\LPA}{\textit{LPA}}
\newcommand{\Hyb}{\textit{Hybrid Louvain-LPA}}
\newcommand{\Sta}{\textit{Static}}
\newcommand{\Nai}{P-ND}
\newcommand{\DelOrg}{\textit{$\Delta$-screening}}
\newcommand{\Del}{P-DDS}
\newcommand{\Fro}{P-DF}
\newcommand{\StaLou}{\textit{Static Louvain}}
\newcommand{\NaiLou}{$\text{P-ND}_\text{L}$}
\newcommand{\DelLou}{$\text{P-DDS}_\text{L}$}
\newcommand{\FroLou}{$\text{P-DF}_\text{L}$}
\newcommand{\StaLPA}{\textit{Static LPA}}
\newcommand{\NaiLPA}{$\text{P-ND}_\text{LPA}$}
\newcommand{\DelLPA}{$\text{P-DDS}_\text{LPA}$}
\newcommand{\FroLPA}{$\text{P-DF}_\text{LPA}$}
\newcommand{\FroHyb}{$\text{P-DF}_\text{H}$}




\begin{document}

%% Full title of the paper.
\title[GVEL: Fast Graph Loading in Edgelist and Compressed Sparse Row (CSR) formats]{GVEL: Fast Graph Loading in Edgelist and\\ Compressed Sparse Row (CSR) formats}

%% Short title to be used in page headers (optional).
% \title[short title]{full title}
% \subtitle{Something other than the title}

%% Authors and their affiliations.
\author{Subhajit Sahu}
\email{subhajit.sahu@research.iiit.ac.in}
\affiliation{%
  \institution{IIIT Hyderabad}
  \streetaddress{Professor CR Rao Rd, Gachibowli}
  \city{Hyderabad}
  \state{Telangana}
  \country{India}
  \postcode{500032}
}

%% Concise author list in page headers.
%\renewcommand{\shortauthors}{Sahu, Kothapalli, and Banerjee, et al.}

%% Show page numbers.
\settopmatter{printfolios=true}

%% Short summary of the work to be presented in the article.
\begin{abstract}
Efficient IO techniques are crucial in high-performance graph processing frameworks like Gunrock and Hornet, as fast graph loading can help minimize processing time and reduce system/cloud usage charges. This research study presents approaches for efficiently reading an Edgelist from a text file and converting it to a Compressed Sparse Row (CSR) representation. On a server with dual 16-core Intel Xeon Gold 6226R processors and Seagate Exos 10e2400 HDDs, our approach, which we term as \textit{GVEL}, outperforms Hornet, Gunrock, and PIGO by significant margins in CSR reading, exhibiting an average speedup of $78\times$, $112\times$, and $1.8\times$, respectively. For Edgelist reading, GVEL is $2.6\times$ faster than PIGO on average, and achieves a Edgelist read rate of $1.9$ billion edges/s. For every doubling of threads, GVEL improves performance at an average rate of $1.9\times$ and $1.7\times$ for reading Edgelist and reading CSR respectively.
\end{abstract}

%% The code below is generated by the tool at http://dl.acm.org/ccs.cfm.
\begin{CCSXML}
<ccs2012>
<concept>
<concept_id>10003752.10003809.10003635</concept_id>
<concept_desc>Theory of computation~Graph algorithms analysis</concept_desc>
<concept_significance>500</concept_significance>
</concept>
</ccs2012>
\end{CCSXML}

% \ccsdesc[500]{Theory of computation~Graph algorithms analysis}

%% Pick words that accurately describe the work being presented.
\keywords{Memory-mapped IO, Parallel Edgelist reading, Parallel CSR reading}

% \received{20 February 2007}
% \received[revised]{12 March 2009}
% \received[accepted]{5 June 2009}



%% Process the author and title information.
\maketitle

\section{Introduction}
\label{sec:introduction}
Graphs are fundamental data structures in many applications, such as computer networks, recommendation systems, and circuit design. In recent years, a number of high performance graph processing frameworks have emerged. State-of-the-art frameworks include Gunrock \cite{wang2016gunrock}, Hornet \cite{busato2018hornet}, Ligra \cite{shun2013ligra}, and Galois \cite{nguyen2013lightweight}.\ignore{These frameworks have demonstrated efficient computation on billion-scale graphs.} Their emphasis lies in accelerating graph analytics tasks by providing high-performance kernels tailored to diverse datasets.

Unfortunately, loading graph data is a significant bottleneck in such frameworks. In fact, the cost of loading data can dominate the overall processing time --- especially as computational capabilities continue to improve\ignore{\cite{gabert2021pigo}}. Gabert and Çatalyürek \cite{gabert2021pigo} observe that\ignore{even on high-performance shared-memory graph systems running billion-scale graphs,} reading the graph from file systems, on such frameworks, takes multiple orders of magnitude longer than running the computational kernel. This slowdown not only causes a disconnect for end users and a loss of productivity for researchers\ignore{/developers}, but also increases the system/cloud usage charges. Fast loading of graphs is thus, crucial\ignore{for minimizing the time it takes to start processing and analyzing the graph data}.\ignore{This motivates us to work on efficient IO techniques. These not only improve response time, but also help lower system / cloud usage charges.}

In modern frameworks like Gunrock, loading graph data from ASCII-based file formats, specifically using the Coordinate (COO) format, is a major bottleneck. To load the graph as an Edgelist, these frameworks typically follow a sequential process of opening the input file, reading the entries one by one, and inserting them into an array. If the goal is to access the graph in the Compressed Sparse Row (CSR) format, which is often the case --- due to its storage efficiency and locality benefits, additional steps are required. These include computing the out-degrees of vertices from the Edgelist, performing prefix sum to determine the offsets of outgoing edges in the CSR representation, and then populating the CSR arrays with edges from the Edgelist. All these operations are carried out sequentially, contributing to the overall loading time.

Many graph processing frameworks\ignore{have showcased efficient computation on large-scale graphs, they}, thus, still rely on sequential I/O. This is likely due to the belief that I/O devices tend to be slow (relative to the CPU), and that achieving parallel I/O necessitates specialized systems.\ignore{Graph and matrix I/O times are seldom reported in the literature.} However, modern IO devices are fast, and implementing only sequential I/O fails to exploit the capabilities of modern Hard Disk Drives (HDDs), Redundant Array of Independent Disks (RAID) controllers, and Non-Volatile Memory (NVM) \cite{gabert2021pigo}. A number of disk-based out-of-memory graph processing systems/frameworks\ignore{\cite{zhu2015gridgraph, cheng2015venus, chi2016nxgraph, ai2017squeezing, ma2017garaph, maass2017mosaic, wu2018redio, ai2018clip, jun2018grafboost, zhang2018wonderland}} \cite{kyrola2012graphchi, han2013turbograph, roy2013x, najeebullah2014bishard, lin2014mmap, zheng2015flashgraph, wang2021scaleg} focus on loading large graphs stored in binary formats. However, a majority of graph datasets exist in serialized human-readable data exchange formats.\ignore{To address this, our focus lies on efficiently loading graphs stored in plain text formats.}

To address these challenges, Gabert and Çatalyürek introduce PIGO \cite{gabert2021pigo}, a header-only, dependency-free C++11 parallel graph loader that supports loading graphs in memory as Edgelists or CSR. PIGO leverages memory mapping, a mechanism that maps a file or part of a file into the virtual memory space\ignore{so that files on the disk can be accessed as if they were in memory} \cite{lin2014mmap}, to optimize file reading. This eliminates the need for repeated system calls, resulting in reduced context-switch overhead and improved efficiency, particularly if the kernel\ignore{accurately} predicts the accessed pages ahead of time.

However, we have identified a few issues with PIGO. Firstly, when reading entries from the input file, PIGO divides the file length equally among threads, potentially leading to slower overall performance as faster threads wait for slower ones. Secondly, PIGO utilizes a two-pass approach for loading graphs into memory as Edgelists, which involves first counting newlines to determine the number of edges and associated offsets for each thread, and then parsing and populating the Edgelist. This method is less efficient compared to a single-pass approach. This method is inefficient compared to a single-pass approach. When converting the Edgelist to a CSR representation, PIGO globally computes vertex degrees using atomics, uses it compute the offsets array of the global CSR, and iterates through the Edgelist to atomically populate the targets array of the global CSR. This global computation of vertex degrees, and directly operating on the shared CSR can lead to high contention between threads. Further, PIGO populates the targets array of the CSR with static load balancing, potentially leading to load imbalances among threads. Finally, when reading Matrix Market (MTX) files, a format commonly used for storing sparse graphs/matrices, PIGO disregards specified attributes, resulting in lower reported runtimes for symmetric graphs (the authors plan to address this\ignore{in the future}).

In this technical report, we propose GVEL\footnote{\url{https://github.com/puzzlef/graph-csr-openmp}}. Similar to PIGO, it employs memory mapping and parallelization to optimize graph loading. However, GVEL improves upon PIGO by efficiently processing the graph as per-thread Edgelists in a single pass through overallocation of memory via memory mapping. Note that this does not waste memory, as untouched pages are never mapped to DRAM. To convert the per-thead Edgelists to CSR, GVEL computes four independent sets of vertex degrees (which, when summed up for each vertex, represents the global degree of each vertex), and uses it to generate the global CSR representation in a novel staged manner. It does this by first obtaining $4$ independent sets of CSRs, and then combining them together, in parallel, to form a global CSR. This minimizes the contention between threads\ignore{, which we observe to be a significant bottleneck for converting an Edgelist to CSR}. These techniques allow GVEL to achieve a $2.6\times$ speedup over PIGO for loading graphs into memory as Edgelists, and a speedup of $1.8\times$ for loading graphs into memory as CSRs (i.e., reading the graph as Edgelist and then converting it to CSR). Our techniques may also be used to convert in-memory Edgelists (an update friendly data structure), to a CSR (a space efficient and locality efficient data structure).
\ignore{Our techniques may also be useful for converting COO to CSR on the fly on parallel devices. This is important since CSR is an efficient data structure, while COO is easy to update. Per-thread COOs are not needed - we can use single COO and split it unto parts for each thread to process.}




% x Why do HP G frameworks load graphs slowly?
% x A quick glimpse of how they do it.
% x Why do the persist with sequential IO? (Modern IO is fast)
% x What are the HP IO interfaces? (MMAP)
% x Which graphs frameworks make use of mmap? (external memory frameworks)
% x What do they focus on? (binary graph formats)
% x Why is fast loading of serialized formats important? (human readable data exchange format)
% x What have Gabert et al. done in PIGO?
% x How do we improve upon it? (link to code)

% - Measure EL, CSR, EL + CSR ...
% - Details of NVME, as much as possible.
% - Indirect comparison with Ligra, GAPbs, and Galois (PIGO).
% - Details of other graphs processing frameworks, and the above
% -- cugraph
% -- networkit
% -- igraph?
% -- gunrock
% -- hornet
% -- graphblast

% - adjust csr partitions
% - adjust csr partitions (convert csr only)
% - adjust block size
% - read EL, convert CSR split on graphs

% - Why fast graph loading is important?
% - Extremely high cost of loading compared to computation.
% - An issue with even popular graph processing frameworks.
% - Modern IO is fast (compared to CPU performance).
% - Storage capacities increasing, bandwidth is high, CPUs as not as fast as they used to be.
% - Common graph file formats (COO, MTX).
% - Common memory storage formats (Edgelist, CSR).
% - Work presented in this paper.




%% - Use --- for a dash.
%% - Use ``camera-ready'' for quotes.
%% - Use {\itshape very} or \textit{very} for italicized text.
%% - Use \verb|acmart| or {\verb|acmart|} for mono-spaced text.
%% - Use \url{https://capitalizemytitle.com/} for URLs.
%% - Use {\bfseries Do not modify this document.} for important boldface details.
%% - Use \ref{fig:name} for referencing.

%% For a block of pre-formatted text: 
% \begin{verbatim}
%   \renewcommand{\shortauthors}{McCartney, et al.}
% \end{verbatim}

%% For a list of items:
% \begin{itemize}
% \item the ``ACM Reference Format'' text on the first page.
% \item the ``rights management'' text on the first page.
% \item the conference information in the page header(s).
% \end{itemize}

%% For a table:
% \begin{table}
%   \caption{Frequency of Special Characters}
%   \label{tab:freq}
%   \begin{tabular}{ccl}
%     \toprule
%     Non-English or Math&Frequency&Comments\\
%     \midrule
%     \O & 1 in 1,000& For Swedish names\\
%     $\pi$ & 1 in 5& Common in math\\
%     \$ & 4 in 5 & Used in business\\
%     $\Psi^2_1$ & 1 in 40,000& Unexplained usage\\
%   \bottomrule
% \end{tabular}
% \end{table}

%% For a full-width table:
% \begin{table*}
%   \caption{Some Typical Commands}
%   \label{tab:commands}
%   \begin{tabular}{ccl}
%     \toprule
%     Command &A Number & Comments\\
%     \midrule
%     \texttt{{\char'134}author} & 100& Author \\
%     \texttt{{\char'134}table}& 300 & For tables\\
%     \texttt{{\char'134}table*}& 400& For wider tables\\
%     \bottomrule
%   \end{tabular}
% \end{table*}


%% For inline math:
% \begin{math}
%   \lim_{n\rightarrow \infty}x=0
% \end{math},

%% For a numbered equation:
% \begin{equation}
%   \lim_{n\rightarrow \infty}x=0
% \end{equation}

%% For an unnumbered equation:
% \begin{displaymath}
%   \sum_{i=0}^{\infty} x + 1
% \end{displaymath}

%% For a figure:
% \begin{figure}[h]
%   \centering
%   \includegraphics[width=\linewidth]{inc/sample-franklin}
%   \caption{1907 Franklin Model D roadster. Photograph by Harris \&
%     Ewing, Inc. [Public domain], via Wikimedia
%     Commons. (\url{https://goo.gl/VLCRBB}).}
%   \Description{A woman and a girl in white dresses sit in an open car.}
% \end{figure}

%% For a teaser figure.
% \begin{teaserfigure}
%   \includegraphics[width=\textwidth]{sampleteaser}
%   \caption{figure caption}
%   \Description{figure description}
% \end{teaserfigure}


\section{Related work}
\label{sec:related}
Graph processing frameworks are software systems designed to efficiently analyze and manipulate graph-structured data, facilitating tasks such as traversal, analytics, and algorithm implementation on various computational architectures. Ligra \cite{shun2013ligra} is a lightweight framework designed for shared-memory parallel/multicore machines, simplifying the development of graph traversal algorithms with two simple routines for edge and vertex mapping. The Galois system \cite{nguyen2013lightweight} advocates for a general-purpose infrastructure supporting fine-grain tasks, speculative execution, and application-specific task scheduling control. Gunrock \cite{wang2016gunrock} tackles irregular data access and control flow issues on GPUs by introducing a data-centric abstraction focused on vertex or edge frontier operations, and enables rapid development of new graph primitives with minimal GPU programming knowledge. Hornet \cite{busato2018hornet} provides a scalable GPU implementation without requiring data reallocation during evolution, targeting dynamic data problems. These frameworks contribute technically to enhancing the efficiency and programmability of graph processing across different computational architectures. While these frameworks have demonstrated efficient computation on billion-scale graphs, they persist with sequential I/O and do not fully exploit the capabilities of modern Hard Disk Drives (HDDs), Redundant Array of Independent Disks (RAID) controllers, Non-Volatile Memory (NVM), and file system caches \cite{gabert2021pigo}.

Incorporation of Memory-Mapped I/O holds promise for improving graph loading efficiency of frameworks. A large number of works have addressed 



In the domain of graph processing frameworks, a range of studies has been conducted to enhance performance and efficiency. Papagiannis et al. \cite{papagiannis2021memory} introduce Aquila, a library OS facilitating customizable memory-mapped I/O (mmio) paths for files or storage devices, demonstrating notable advantages in key-value stores and graph processing applications. Leis et al. \cite{leis2023virtual} propose vmcache, a buffer manager design that utilizes hardware-supported virtual memory for efficient page identifier translation to virtual memory addresses, offering flexible, efficient, and scalable buffer management on multi-core CPUs and fast storage devices. Feng et al. \cite{feng2023tricache} present TriCache, a cache mechanism enabling in-memory programs to process out-of-core datasets efficiently through a multi-level block cache design. Additionally, Song et al. \cite{song2012low} investigate the Linux virtual memory subsystem and mmap() I/O path for low-latency storage devices, proposing optimization policies to mitigate overheads. Imamura and Yoshida \cite{imamura2019poster} propose AR-MMAP, an asynchronous read method reducing write latency in memory-mapped files. Li et al. \cite{li2019userland} propose CO-PAGER, a lightweight userspace memory service offering customizable and efficient memory paging. These studies collectively contribute to the optimization and advancement of graph processing frameworks, addressing challenges and enhancing performance in various computational contexts.

%%
The literature on memory-mapped I/O (mmap) optimization in the context of high-performance computing (HPC) reveals various approaches to enhance graph loading speed. Song et al. \cite{song2016efficient} demonstrate the effectiveness of mmap with fast storage devices in Linux, while Malliotakis et al. \cite{malliotakis2021hugemap} introduce HugeMap, a custom mmap path using huge pages for improved throughput and reduced system time.

Papagiannis et al. \cite{papagiannis2020optimizing} propose FastMap, an mmap alternative designed for multi-core servers, reducing synchronization overhead and increasing device queue depth. Our experiments show FastMap's scalability to 80 cores, providing significant IOPS and throughput improvements compared to traditional mmap, especially with Optane SSDs.

Essen et al. \cite{van2015di} present DI-MMAP, a high-performance runtime that outperforms Linux mmap in memory-mapping large external datasets. Papagiannis et al. \cite{papagiannis2021memory} contribute Aquila, a library OS offering mmio compatibility, DRAM I/O cache customization, and reduced I/O overhead. Aquila demonstrates substantial benefits in key-value stores and graph processing applications, outperforming Linux mmap in both cache management and execution time.

Alverti et al. \cite{alverti2022daxvm} propose DaxVM, extending OS virtual memory and file system layers for a fast and scalable DAX-mmap interface. DaxVM achieves higher throughput than default mmap for the Apache webserver and PMem-optimized key-value stores, showcasing its relevance beyond byte-addressable storage.

Song et al. \cite{song2012low} investigate the Linux virtual memory subsystem and mmap() I/O path, proposing optimizations to reduce mmap() I/O overhead. Their solution ensures faster mmio compared to read-write I/O under high cache-hit ratios.
Imamura and Yoshida \cite{imamura2019poster} present AR-MMAP, an asynchronous read method to reduce write latency in memory-mapped files. Li et al. \cite{li2019userland} propose CO-PAGER, a userspace memory service, bypassing kernel I/O stacks for efficient memory paging. Feng et al. \cite{feng2023tricache} introduce TriCache, a cache mechanism for in-memory programs to process out-of-core datasets without code modification.

Leis et al. \cite{leis2023virtual} propose vmcache, a buffer manager using hardware-supported virtual memory for page translation, ensuring flexible, efficient, and scalable buffer management on multi-core CPUs and fast storage devices. Their second contribution, exmap, addresses performance bottlenecks in page table manipulation on Linux, collectively providing a comprehensive solution.
%%




%% MMAP
Song et al. \cite{song2012low} examine linux virtual memory subsystem and mmap() I/O path to figure out the influence of low-latency storage devices on the existing virtual memory subsystem. Also, we suggest some optimization policies to reduce the overheads of mmap() I/O and implement the prototype in a recent Linux kernel. Our solution guarantees that 1) memory-mapped I/O will be several times faster than read-write I/O when cache-hit ratio becomes high, and 2) the former will show at least the performance of the latter even when cache-miss frequently occurs and the overhead of mapping/unmapping pages becomes significant, which are not achievable by the existing virtual memory subsystem.

%% MMAP
Essen et al. \cite{van2015di} present DI-MMAP, a high-performance runtime that memory-maps large external data sets into an application’s address space and shows better performance than the Linux mmap system call.

%% MMAP
Song et al. \cite{song2016efficient} optimize Linux's memory mapped IO (mmio) for fast storage devices. Their results indicate that mmap can be used effectively with fast storage.

%% MMAP
Malliotakis et al. \cite{malliotakis2021hugemap} present HugeMap, a custom mmio path in the Linux kernel that uses huge pages for file-backed mappings to accelerate applications with sequential I/O access patterns or large I/O operations. Their experiments show up to higher throughput and lower system time, compared to regular page configurations.

%% MMAP
Imamura and Yoshida \cite{imamura2019poster} propose AR-MMAP that is an asynchronous read method to reduce the write latency of applications applying memory-mapped files. It hides the read latency of block devices by reading the corresponding pages from block devices in background. As AR-MMAP requires the assistance of applications to guarantee data consistency, we modify an in-memory key-value store as a use case. We implement AR-MMAP in a Linux kernel and evaluate its performance on a server containing two types of SSDs. The evaluation results demonstrate that it reduces the execution time compared to a default kernel.

%% MMAP
Li et al. \cite{li2019userland} propose CO-PAGER, which is a lightweight userspace memory service. CO-PAGER consists of a minimal kernel module and a userspace component. The userspace component handles (redirected) page faults, performs memory management and I/O operations and accesses NVM storage directly. The kernel module is used to update memory mapping between user and kernel space. In this way CO-PAGER can bypass the deep kernel I/O stacks and provide a flexible/customizable and efficient memory paging service in userspace. We provide a general programming interface to use the CO-PAGER service. In our experiments, we also demonstrate how the CO-PAGER approach can be applied to a MapReduce framework and improves performance for data-intensive applications.

%% MMAP
Papagiannis et al. \cite{papagiannis2020optimizing} propose FastMap, an alternative design for the memory-mapped I/O path in Linux that provides scalable access to fast storage devices in multi-core servers, by reducing synchronization overhead in the common path. FastMap also increases device queue depth, an important factor to achieve peak device throughput. Our experimental analysis shows that FastMap scales up to 80 cores and provides up to 11.8× more IOPS compared to mmap using null\_blk. Additionally, it provides up to 5.27x higher throughput using an Optane SSD. We also show that FastMap is able to saturate state-of-the-art fast storage devices when used by a large number of cores, where Linux mmap fails to scale.

%% MMAP
Papagiannis et al. \cite{papagiannis2021memory} present Aquila, a library OS that allows applications to reduce I/O overhead by customizing the memory-mapped I/O (mmio) path for files or storage devices. Compared to Linux mmap, Aquila (a) offers full mmio compatibility and protection to minimize application modifications, (b) allows applications to customize the DRAM I/O cache, its policies, and access to storage devices, and (c) significantly reduces I/O overhead. Aquila achieves its mmio compatibility, flexibility, and performance by placing the application in a privileged domain, non-root ring 0. We show the benefits of Aquila in two cases: (a) Using mmio in key-value stores to reduce I/O overhead and (b) utilizing mmio in graph processing applications to extend the memory heap over fast storage devices. Aquila requires 2.58× fewer CPU cycles for cache management in RocksDB, compared to user-space caching and read/write system calls and results in 40\% improvement in request throughput. Finally, we use Ligra, a graph processing framework, to show the efficiency of Aquila in extending the memory heap over fast storage devices. In this case, Aquila results in up to 4.14× lower execution time compared to Linux mmap.

%% MMAP
Alverti et al. \cite{alverti2022daxvm} propose DaxVM, a design that extends the OS virtual memory and file system layers leveraging persistent memory attributes to provide a fast and scalable DAX-mmap interface. DaxVM eliminates paging costs through pre-populated file page tables, supports faster and scalable virtual address space management for ephemeral mappings, performs unmappings asynchronously, bypasses kernel-space dirty-page tracking support, and adopts asynchronous block pre-zeroing. We implement DaxVM in Linux and the ext4 file system targeting xS6-64 architecture. DaxVM mmap achieves 4.9x higher throughput than default mmap for the Apache webserver and up to 1.5x better performance than read system calls. It provides similar benefits for text search. It also provides fast boot times and up to 2.95x better throughput than default mmap for PMem-optimized key-value stores running on a fragmented ext4 image. Despite designed for direct access to byte-addressable storage, various aspects of DaxVM are relevant for efficient access to other high performant storage mediums.

%% MMAP
Feng et al. \cite{feng2023tricache} propose TriCache, a cache mechanism that enables in-memory programs to efficiently process out-of-core datasets without requiring any code rewrite. It provides a virtual memory interface on top of the conventional block interface to simultaneously achieve user transparency and sufficient out-of-core performance. A multi-level block cache design is proposed to address the challenge of per-access address translations required by a memory interface. It can exploit spatial and temporal localities in memory or storage accesses to render storage-to-memory address translation and page-level concurrency control adequately efficient for the virtual memory interface.

%% MMAP
Leis et al. \cite{leis2023virtual} propose vmcache, a buffer manager design that instead uses hardware-supported virtual memory to translate page identifiers to virtual memory addresses. In contrast to existing mmap-based approaches, the DBMS retains control over page faulting and eviction. Our design is portable across modern operating systems, supports arbitrary graph data, enables variable-sized pages, and is easy to implement. One downside of relying on virtual memory is that with fast storage devices the existing operating system primitives for manipulating the page table can become a performance bottleneck. As a second contribution, we therefore propose exmap, which implements scalable page table manipulation on Linux. Together, vmcache and exmap provide flexible, efficient, and scalable buffer management on multi-core CPUs and fast storage devices.




%% MMAP LOAD
Graph and sparse matrix systems are highly tuned, able to run complex graph analytics in fractions of seconds on billion-edge graphs. For both developers and researchers, the focus has been on computational kernels and not end-to-end runtime. Despite the significant improvements that modern hardware and operating systems have made towards input and output, these can still become application bottlenecks. Unfortunately, on high-performance shared-memory graph systems running billion-scale graphs, reading the graph from file systems easily takes over 2000x longer than running the computational kernel. This slowdown causes both a disconnect for end users and a loss of productivity for researchers and developers. Gabert and Çatalyürek \cite{gabert2021pigo} close the gap by providing a simple to use, small, header-only, and dependency-free C++11 library that brings I/O improvements to graph and matrix systems. Using our library, we improve the end-to-end performance for state-of-the-art systems significantly-in many cases by over 40x.

%% MMAP GRAPH
Lin et al. \cite{lin2014mmap} use the fundamental memory mapping capability of all modern hardware to create fast and scalable graph algorithms. They are able to to process 6.6 billion edge Yahoo web graph faster than other graph processing frameworks, such as TurboGraph and GraphChi, with reduced complexity - both in terms of simpler data structures, and fewer lines of code. They benefit from existing page replacement policies, such as LRU.

%% MMAP GRAPH
Wang et al. \cite{wang2019lgraph} present LGraph, a unified data model and API for productive open-source hardware design. Key features of LGraph include a
unified data model and API, a fast memory mapped library design, integration with third-party tools and hierarchical design traversal for third-party tools.

%% MMAP GRAPH
Kim and Swanson \cite{kim2022blaze} introduce Blaze, a new out-of-core graph processing system optimized for ultra-low-latency SSDs. Blaze offers high-performance out-of-core graph analytics by constantly saturating these fast SSDs with a new scatter-gather technique called online binning that allows value propagation among graph vertices without atomic synchronization. Blaze offers succinct APIs to allow programmers to write efficient out-of-core graph algorithms without the burden to manage complex IO executions. Our evaluation shows that Blaze outperforms current out-of-core systems by a wide margin on seven datasets and a set of representative graph queries on Intel Optane SSD.

%% MMAP GRAPH
Han et al. \cite{han2013turbograph} propose a general, disk-based graph engine called TurboGraph to process billion-scale graphs very efficiently by using modern hardware on a single PC. TurboGraph is the first truly parallel graph engine that exploits 1) full parallelism including multi-core parallelism and FlashSSD IO parallelism and 2) full overlap of CPU processing and I/O processing as much as possible. Specifically, we propose a novel parallel execution model, called pin-and-slide. TurboGraph also provides engine-level operators such as BFS which are implemented under the pin-and-slide model. Extensive experimental results with large real datasets show that TurboGraph consistently and significantly outperforms Graph-Chi by up to four orders of magnitude! Our implementation of TurboGraph is available at ``http://wshan.net/turbograph" as executable files.

%% MMAP GRAPH
Zhou and Hoffmann \cite{zhou2018graphz} present two innovations that improve the performance of software frameworks for out-of-core graph analytics. The first is degree-ordered storage, a new storage format that dramatically lowers book-keeping overhead when graphs are larger than memory. The second innovation replaces existing static messages with novel ordered dynamic messages which update their destination immediately, reducing both the memory required for intermediate storage and IO pressure. We implement these innovations in a framework called GraphZ-which we release as open source-and we compare its performance to two state-of-the-art out-of-core graph frameworks. For graphs that exceed memory size, GraphZ's harmonic mean performance improvements are 1.8-8.3× over existing solutions.

%% MMAP GRAPH
Liu and Huang \cite{liu2017graphene} strive to achieve an ambitious goal of achieving ease of programming and high IO performance (as in-memory processing) while maintaining graph data on disks (as external memory processing). To this end, we have designed and developed Graphene that consists of four new techniques: an IO request centric programming model, bitmap based asynchronous IO, direct hugepage support, and data and workload balancing. The evaluation shows that Graphene can not only run several times faster than several external-memory processing systems, but also performs comparably with in-memory processing on large graphs.


\section{Preliminaries}
\label{sec:preliminaries}
\subsection{Graph Storage and In-Memory Formats}

Graphs can be stored in either text or binary formats. Text formats, such as edgelists, provide a simple representation where each line denotes an edge between two vertices (optionally including edge weights and other attributes). While text formats are space-inefficient compared to binary formats, their readability and ease of sharing contribute to their widespread adoption in the public domain. Matrix Market Format (MTX) is a standardized text format with headers specifying the matrix's properties, making it suitable for various sparse matrix representations.

In-memory graph formats, on the other hand, optimize for efficient data access during computation. These include edgelists, which mirror the structure of the edgelist storage format. To improve data locality, the source vertex, target vertex, and edge weights may be stored in separate arrays. This format is easily convertible from edgelist storage, and is suitable for graph algorithms requiring edge-oriented computation. Compressed Sparse Row (CSR) is another popular in-memory format that is optimal for vertex-oriented algorithms, such as traversal. It stores graph data in three arrays: one for offsets to the list of target vertices / edge weights of each vertex, one for target vertices (for each vertex), and one for edge weights (for each vertex).




\subsection{Memory-Mapped I/O}

Memory-mapped I/O (MMIO) is an access method that maps files or file-like resources to a memory region, providing applications with data access through memory semantics. \texttt{mmap()}, a key system call in memory-mapped I/O, is used to map files into memory, establishing a virtual memory mapping between the process's address space and the file or device. When a thread accesses a page that has not yet been loaded from the file, a page fault occurs, and the thread is put into sleep by the kernel until data from the file has been loaded into the page. The \texttt{madvise()} system call allows programmers to advise the kernel about their expected access patterns for the mapped region, optimizing performance. Subsequently, \texttt{munmap()} is employed to unmap the region, freeing up resources.

MMIO has reduced overhead and eliminates copies between kernel and user space \cite{malliotakis2021hugemap, papagiannis2020optimizing}. Hot data typically resides in main memory, leveraging the benefits of a large cache, while cold data are stored on devices such as HDDs and SSDs \cite{song2016efficient}. The increasing adoption of MMIO is driven by its superior performance, particularly in comparison to file semantics like read/write operations which introduces significant overhead for fast storage \cite{yoshimura2019evfs, enberg2022transcending}. 




\subsection{Stream Operations in C++}

The Standard Template Library (STL) in C++ offers a suite of stream classes and methods. C++ stream classes include \texttt{ifstream} for input file streams and \texttt{ofstream} for output file streams. The \texttt{istream} and \texttt{ostream} interfaces provide a common set of methods for reading and writing data. The \texttt{>>} and \texttt{<<} operators are overloaded for various data types and provide read and writing of data from and to streams, while \texttt{getline()} method reads a line from the stream. C++ streams use an internal buffer to optimize I/O operations.




\subsection{Number Parsing Methods in C++}

C++ provides several methods for converting textual data to numerical representations. \texttt{sscanf()} is a powerful method for parsing formatted input strings. It operates similarly to `printf` and allows developers to define a format specifier that describes the expected structure of the input string. On the other hand, \texttt{strtoull()} and \texttt{strtod()} offer a simple and direct approach for converting strings to unsigned long long integers and double-precision floating-point numbers, respectively. Performance-wise, \texttt{sscanf()} might involve a bit more overhead due to the need to interpret format specifiers, while \texttt{strtoull} and \texttt{strtod} provide straightforward conversions without the need for format strings.


\section{Approach}
\label{sec:approach}
\subsection{Reading Edgelist from text file}

We attempt a number of approaches to read edgelist from text file into in-memory edgelist(s), given in Sections \ref{sec:el-fstream-plain}-\ref{sec:el-mmap-custom}. Among these, we find using \texttt{mmap()} with custom number parsers (\textit{mmap-custom}) to be the best approach. The pseudocode for \textit{mmap-custom} is given Algorithm \ref{alg:el}.

\begin{algorithm}[hbtp]
\caption{Reading Edge-list from file.}
\label{alg:el}
\begin{algorithmic}[1]
\Require{$pdegrees$: Per partition vertex degrees (output)}
\Require{$edges$: Per thread sources, targets, and weights of edges (output)}
\Require{$data$: Memory mapped file data}
\Ensure{$counts$: Number of edges read per thread (output)}
\Ensure{$symmetric$: Is graph symmetric?}
\Ensure{$weighted$: Is graph weighted?}
\Ensure{$\beta$: Size of each block that is processed per thread}
\Ensure{$\rho$: Number of partitions for counting vertex degrees}
\Ensure{$t$: Current thread}

\Statex

\Function{getBlock}{$data, i$} \label{alg:frontier--main-begin}
  \State $[d, D] \gets data$
  \State $b \gets d+i$ \textbf{;} $B \gets min(b+\beta, D)$
  \If{$b \neq d$ \textbf{and not} $isNewline(b-1)$}
    \State $b \gets findNextLine(b, D)$
  \EndIf
  \If{$B \neq d$ \textbf{and not} $isNewline(B-1)$}
    \State $B \gets findNextLine(B, D)$
  \EndIf
  \Return{$[b, B]$}
\EndFunction

\Statex
  
\Function{readEdgelist}{$pdegrees, edges, data$}
  \State $counts \gets \{0\}$
  \State $[sources, targets, weights] \gets edges$
  \State $\rhd$ Load edges from text file in blocks of size $\beta$
  \ForAll{$i \in [0, \beta, 2\beta, ... |data|]$ \textbf{in parallel}}
    \State $j \gets counts[t]$
    \State $[b, B] \gets getBlock(data, i)$
    \While{$true$}
      \State $\rhd$ Read an edge from the block
      \State $u \gets v \gets 0$ \textbf{;} $w \gets 1$
      \State $b \gets findNextDigit(b, B)$
      \If{$b = B$} \textbf{break}
      \EndIf
      \State $b \gets parseWholeNumber(u, b, B)$
      \State $b \gets findNextDigit(b, B)$
      \State $b \gets parseWholeNumber(v, b, B)$
      \If{$weighted$}
        \State $b \gets findNextDigit(b, B)$
        \State $b \gets parseFloat(w, b, B)$
      \EndIf
      \State $\rhd$ Make it zero-based
      \State $u \gets u - 1$ \textbf{;} $v \gets v - 1$
      \State $\rhd$ Add the parsed edge to edgelist
      \State $sources[t][j] \gets u$
      \State $targets[t][j] \gets v$
      \If{$weighted$} $weights[t][j] \gets w$
      \EndIf
      \State $atomicAdd(pdegrees[t \bmod \rho][u], 1)$
      \State $j \gets j + 1$
      \State $\rhd$ If graph is symmetric, add the reverse edge
      \If{$symmetric$}
        \State $sources[t][j] \gets v$
        \State $targets[t][j] \gets u$
        \If{$weighted$} $weights[t][j] \gets w$
        \EndIf
        \State $atomicAdd(pdegrees[t \bmod \rho][v], 1)$
        \State $j \gets j + 1$
      \EndIf
    \EndWhile
    \State $counts[t] \gets j$
  \EndFor
  \Return{$counts$}
\EndFunction \label{alg:frontier--main-end}
\end{algorithmic}
\end{algorithm}




%% Parameter setting
% TOLERANCE = 0.05
% MAX\_ITERATIONS = 20
% MAX\_THREADS = 12



\subsubsection{\texttt{ifstream} with \texttt{getline()} and \texttt{>>} operator (\textit{fstream-plain})}
\label{sec:el-fstream-plain}

In this approach, we utilize C++'s \texttt{ifstream} to open the file, and read the edges line by line with \texttt{getline()}. If the graph is unweighted, we read the source and target vertex ids as 64-bit unsigned integers, using \texttt{>>} operator, into pre-allocated source and target arrays based on information in the file header. If the graph is weighted, we also read the weights as 32-bit floating-point numbers into another pre-allocated array. This process is sequential, given that streams are inherently sequential.


\subsubsection{\texttt{ifstream} with \texttt{getline()} and \texttt{strto*()} (\textit{fstream-strto*})}
\label{sec:el-fstream-stro*}

In this approach, we again use \texttt{ifstream} but employ string-to-number conversion methods \texttt{strtoull()} and \texttt{strtod()} for parallel number parsing. We sequentially read a block of $L$ lines from the file, using \texttt{getline()}, and then parse each line in parallel using multiple threads. We observe that using OpenMP's dynamic scheduling, with a chunk size of $1024$, and reading a block of $L=128K$ lines to be processed in parallel offers the best performance. Parsed edges (source, target vertex ids, and edge weights) are stored separately in per-thread edge lists to avoid contention issues within a shared data structure. With 64 threads, this approach demonstrates a speedup of XX compared to \textit{fstream-plain}, as shown in Figure \ref{fig:optimize-el}.


\subsubsection{\texttt{fopen()} with \texttt{fgets()} and \texttt{sscanf()} (\textit{fopen-plain})}
\label{sec:el-fopen-plain}

This approach is similar to the one mentioned above (\textit{fscanf-strto*}), but we use \texttt{fgets()} on a file handle to read lines instead of \texttt{getline()}, and employ \texttt{sscanf()} to parse the edges. With 64 threads, it provided a speedup of X compared to X.


\subsubsection{\texttt{fopen()} with \texttt{fgets()} and \texttt{strto*()} (\textit{fopen-strto*})}
\label{sec:el-fopen-strto*}

Similar to the previous approach (\textit{fopen-plain}), this one uses \texttt{fgets()} to read lines from the text file, but replaces \texttt{sscanf()} with \texttt{strtoull()} and \texttt{strtod()}. This proves faster due to the absence of a format string. With 64 threads, it provides a speedup of X over X.


\subsubsection{\texttt{mmap()} with \texttt{strto*()} (\textit{mmap-strto*})}
\label{sec:el-mmap-strto*}

In this approach, we map the file to memory with \texttt{mmap()}, and process the edges in parallel by partitioning the file into blocks of $C$ characters. Each block is dynamically assigned (using OpenMP's dynamic schedule) to a free thread. If the assigned block contains partial lines at either end, the thread repositions it, by shifting to the right to eliminate partial lines. This involves skipping the partial line at the beginning and including the partial line from the end. We observe that issuing \texttt{madvice(MADV\_WILLNEED)}, and using a block size of $C=256K$ characters offers the best performance. To parse the source/target vertex ids and edge weights, we use \texttt{strtoull()} and \texttt{strtod()}. Each thread stores the parsed edges in per-thread edgelists.


\subsubsection{\texttt{mmap()} with custom number parsers (\textit{mmap-custom})}
\label{sec:el-mmap-custom}

This is similar to the approach mentioned above (\textit{mmap-strto*}), but we use our own functions for parsing whole numbers and floating-point numbers. In addition, as vertex ids start with $1$, we decrement $1$ from the vertex ids after parsing it and before appending them to per-thread edgelists. Surprisingly, this leads to $40-50\%$ drop in performance. Converting the $weighted$ flag (see Algorithm \ref{alg:el}) to a template parameter solves this issue. This indicates that the issue was related to the loop code not being able to fit in the code cache of the processor and using a template allowed it to fit in the cache. Accordingly, we also recommend using $symmetric$ flag to be used as a template parameter instead. With 64 threads, it provides a speedup of X over X, as show in Figure \ref{fig:optimize-el}. We also attempted to use custom SIMD instructions to parse numbers, along with \texttt{vzeroupper} instruction to clear SSE/AVX registers, but it did not provide additional performance improvement.


\subsubsection{Explanation of \textit{mmap-custom} approach (Algorithm \ref{alg:el})}

We now explain the psuedocode of \textit{mmap-custom} approach, which loads per-thread edgelists from a file with the best performance. Consider the \texttt{readEdgelist()} function. In Lines X-X various variables are initialized, including the counts of edges read per thread ($counts$), arrays for sources, targets, and weights ($edges$), and other parameters. This is followed by a loop (Lines X-X), where each iteration processes a block of characters in the text file in parallel across different threads. The loop iterates over block of $data$, starting from index $i$ with a step size of $\beta = 256K$. Inside the loop, $j$ keeps track of the number of edges processed by the current thread. In Line X, the \texttt{getBlock()} function is called to retrieve the current block of data ($[b, B]$). The algorithm then enters a While loop to read edges from the block in Line X-X. Edge information (source, target, weight) is parsed from the block, and, if the graph is weighted, the weight is also parsed. In Line X-X, Parsed edges are adjusted to be zero-based, and vertex degrees are updated in the pdegrees array. Then, in Lines X-X, the parsed edges are added to the arrays (sources, targets, weights) for the current thread. If the graph is symmetric, reverse edges are added as well. The loop continues until the entire block is processed, updating j and counting the number of edges processed by the current thread. Finally, the number of edges read by each thread are returned.

The \texttt{getBlock()} function (Lines X-X) retrieves a block of characters to process from the memory-mapped file, starting from index $i$. It ensures that the block starts and ends on newline characters for proper parsing. The block size is determined by the parameter $\beta$, which is set to $256K$.

\begin{figure}[hbtp]
  \centering
  \includegraphics[width=0.99\linewidth]{out/optimize-el.pdf} \\[-2ex]
  \caption{Relative runtime of reading per-thread edge-lists using C++'s input file stream (\textit{fstream-plain}), file stream to read lines and using  \texttt{strtoull} and \texttt{strtod} function for parsing numbers (\textit{fstream-strtoull}), \texttt{fgets} to read lines and \texttt{sscanf} for parsing numbers (\textit{fscanf-plain}), \texttt{fgets} with \texttt{strtoull} and \texttt{strtod} (\textit{fscanf-strtoull}), memory mapped file using \texttt{mmap} with \texttt{strtoull} and \texttt{strtod} (\textit{mmap-strtoull}), \texttt{mmap} with custom integer/float parsing functions (\textit{mmap-custom}), and \texttt{mmap} with custom integer/float parsing functions along with making vertex id 0-based (\textit{mmap-base1}).}
  \label{fig:optimize-el}
\end{figure}




\subsection{Converting Edgelist to CSR}

\begin{algorithm}[hbtp]
\caption{Convert Edge-list to CSR.}
\label{alg:csr}
\begin{algorithmic}[1]
\Require{$csr$: Global CSR (output)}
\Require{$pcsr$: Per partition CSR (scratch)}
\Require{$pdegrees$: Per partition vertex degrees (scratch)}
\Require{$edges$: Per thread sources, targets, and weights of edges}
\Require{$counts$: Number of edges read per thread}
\Ensure{$symmetric$: Is graph symmetric?}
\Ensure{$weighted$: Is graph weighted?}
\Ensure{$\rho$: Number of partitions for counting vertex degrees}
\Ensure{$t$: Current thread}

\Statex

\Function{convertToCsr}{$csr, pcsr, pdegrees, edges, counts$}
  \State $[offsets, edgeKeys, edgeValues] \gets csr$
  \State $[poffsets, pedgeKeys, pedgeValues] \gets pcsr$
  \State $[sources, targets, weights] \gets edges$
  \State $\rhd$ Compute offsets
  \ForAll{$p \in [0, \rho)$}
    \State $exclusiveScan(poffsets[p], pdegrees[p], |V|+1)$
  \EndFor
  \State $\rhd$ Populate per-partition CSR
  \ForAll{\textbf{threads in parallel}}
    \ForAll{$i \in [0, counts[t])$}
      \State $u \gets sources[t][i]$
      \State $v \gets targets[t][i]$
      \State $j \gets atomicAdd(poffsets[t \bmod \rho][u], 1)$
      \State $pedgeKeys[t \bmod \rho][j] \gets v$
      \If{$weighted$}
        \State $pedgeValues[t \bmod \rho][j] \gets weights[t][i]$
      \EndIf
    \EndFor
  \EndFor
  \State $\rhd$ Fix per-partition offsets
  \ForAll{\textbf{threads in parallel}}
    \If{$t < \rho$}
      \State $memcpy(poffsets[t]+1, poffsets[t], |V|)$
      \State $poffsets[t][0] \gets 0$
    \EndIf
  \EndFor
  \State $\rhd$ Combine per-partition offsets
  \ForAll{$u \in [0, |V|)$ \textbf{in parallel}}
    \ForAll{$p \in [1, \rho)$}
      \State $pdegrees[0][u] +\gets pdegrees[p][u]$
    \EndFor
  \EndFor
  \State $\rhd$ Compute global offsets
  \State $exclusiveScan(offsets, pdegrees[0], |V|+1)$
  \State $\rhd$ Combine per-partition CSR into one CSR
  \ForAll{$u \in [0, |V|)$ \textbf{in parallel}}
    \State $j \gets offsets[u]$
    \ForAll{$p \in [0, \rho)$}
      \State $i \gets poffsets[t][u]$
      \State $I \gets poffsets[t][u+1]$
      \ForAll{$i \in [i, I)$}
        \State $edgeKeys[j] \gets pedgeKeys[t][i]$
        \If{$weighted$}
          \State $edgeValues[j] \gets pedgeValues[t][i]$
        \EndIf
        \State $j \gets j + 1$
      \EndFor
    \EndFor
  \EndFor
\EndFunction
\end{algorithmic}
\end{algorithm}



\subsubsection{Obtain global vertex degrees along with reading Edgelist (\textit{degree-global})}

Now that we have obtained per-thread edgelists, we must now convert the edgelists to CSR. To do this, we first need to know the degree of each vertex. In this approach, A simple solution for this is to update the degree of each vertex in a common array, using atomic operations, while reading the edgelists. The relative runtime of this approach with respect to simple reading per-thread edgelists is XX, as shown in Figure \ref{fig:optimize-csr}. We however observe that this results in high contention and impacts performance.


\subsubsection{Obtain per-thread vertex degrees along with reading Edgelist (\textit{degree-thread})}

In this approach, we compute per-thread vertex degrees instead of global degrees. While this improves performance by X\% compared to obtaining global vertex degrees, it requires significant additional space, and needs to be combined later to obtain global degrees. We observe that computing degrees in partitions of $4$ (using $\bmod 4$) and then combining them gives us the best performance.


\subsubsection{Obtain CSR from global vertex degrees (\textit{csr-global})}

Now that we have the degrees, we need to combine the per-thread edgelists into a CSR data structure. In this approach, we first obtain global vertex degrees along with read per-thread edgelists (as with \textit{degree-global}), and the convert the per-thread edgelists to a global CSR in parallel using atomic operations. However, this approach has poor performance.


\subsubsection{Obtain CSR from 4-partitioned vertex degrees (\textit{csr-partition4})}

Next, we explore computing CSR in k partitions and later combining them. Our observations indicate that using 4 partitions to generate CSR and then combining them has the best performance.

Now describe the figures and the algorithm.

\begin{figure}[hbtp]
  \centering
  \includegraphics[width=0.99\linewidth]{out/optimize-csr.pdf} \\[-2ex]
  \caption{Relative runtime of reading per-thread edge-lists from file (\textit{edgelist}), reading per-thread edge-lists + global degrees (\textit{degree-global}), reading per-thread edge-lists + per-thread degrees (\textit{degree-thread}), reading per-thread edge-lists + global degrees + converting to global CSR (\textit{csr-global}), and reading per-thread edge-lists + 4-partition degrees + converting to 4-partition CSR + converting to global CSR (\textit{csr-partition-4}).}
  \label{fig:optimize-csr}
\end{figure}



\section{Evaluation}
\label{sec:evaluation}
\subsection{Experimental Setup}
\label{sec:setup}

We use a server that has two $16$-core x86-based Intel Xeon Gold 6226R processors running at $2.90$ GHz. Each core has an L1 cache of $1$ MB, an L2 cache of $16$ MB, and a shared L3 cache of $22$ MB. The machine has $93.4$ GB of system memory and runs on CentOS Stream 8. We use GCC 8.5 and OpenMP 4.5. Table \ref{tab:dataset} shows the graphs we use in our experiments. All of them are obtained from the SuiteSparse Matrix Collection \cite{kolodziej2019suitesparse}.

\begin{table}[!ht]
  \centering
  \caption{In our experiments, we use a list of 13 graphs. The table lists the total number of vertices ($|V|$), total number of edges ($|E|$), and the file size ($F_{size}$) for each graph.}
  \label{tab:dataset}
  \begin{tabular}{|c||c|c|c|}
    \toprule
    \textbf{Graph} &
    \textbf{\textbf{$|V|$}} &
    \textbf{\textbf{$|E|$}} &
    \textbf{\textbf{$F_{size}$}} \\
    \midrule
    \multicolumn{4}{|c|}{\textbf{Web Graphs (LAW)}} \\ \hline
    indochina-2004 & 7.41M & 194M & 2.9 GB \\ \hline
    uk-2002 & 18.5M & 298M & 4.7 GB \\ \hline
    arabic-2005 & 22.7M & 640M & 11 GB \\ \hline
    uk-2005 & 39.5M & 936M & 16 GB \\ \hline
    webbase-2001 & 118M & 1.02B & 18 GB \\ \hline
    it-2004 & 41.3M & 1.15B & 19 GB \\ \hline
    sk-2005 & 50.6M & 1.95B & 33 GB \\ \hline
    \multicolumn{4}{|c|}{\textbf{Social Networks (SNAP)}} \\ \hline
    com-LiveJournal & 4.00M & 34.7M & 480 MB \\ \hline
    com-Orkut & 3.07M & 117M & 1.7 GB \\ \hline
    \multicolumn{4}{|c|}{\textbf{Road Networks (DIMACS10)}} \\ \hline
    asia\_osm & 12.0M & 12.7M & 200 MB \\ \hline
    europe\_osm & 50.9M & 54.1M & 910 MB \\ \hline
    \multicolumn{4}{|c|}{\textbf{Protein k-mer Graphs (GenBank)}} \\ \hline
    kmer\_A2a & 171M & 180M & 3.2 GB \\ \hline
    kmer\_V1r & 214M & 233M & 4.2 GB \\ \hline
  \bottomrule
  \end{tabular}
  \end{table}


\begin{figure*}[hbtp]
  \centering
  \includegraphics[width=0.68\linewidth]{out/runtime.pdf} \\[-2ex]
  \caption{Time taken by GVEL for edge-list and CSR loading on 13 different graphs.}
  \label{fig:runtime}
\end{figure*}

\begin{figure*}[hbtp]
  \centering
  \includegraphics[width=0.68\linewidth]{out/scaling.pdf} \\[-2ex]
  \caption{Speedup of GVEL's Edgelist and CSR loading with increasing number of threads.}
  \label{fig:scaling}
\end{figure*}

\begin{figure*}[hbtp]
  \centering
  \includegraphics[width=0.68\linewidth]{out/compare-large.pdf} \\[-2ex]
  \caption{Time taken by Hornet, Gunrock, PIGO, and GVEL for reading Edgelist and converting it to CSR on 13 different graphs. PIGO and GVEL are not visible on this scale - they are significantly faster than Hornet and Gunrock. The graph loading time for Hornet is not shown for \textit{uk-2002}, \textit{it-2004}, and \textit{sk-2005} graphs as it crashed while loading.}
  \label{fig:compare-large}
\end{figure*}

\begin{figure*}[hbtp]
  \centering
  \includegraphics[width=0.68\linewidth]{out/compare-small.pdf} \\[-2ex]
  \caption{Time taken by PIGO and GVEL for reading Edgelist and converting it to CSR on 13 different graphs.}
  \label{fig:compare-small}
\end{figure*}



In our comparison with Hornet, Gunrock, and PIGO, we observe that:

PIGO has demonstrated superiority over XXX, prompting our exclusion from the comparison.
The average speedup over Gunrock showcases the efficiency of our approach.
Similarly, the average speedup over PIGO further highlights the performance benefits of our method.
Notably, our approach excels on web graphs characterized by power law distributions and high average degrees.
We present our key findings from the observations in the figure captions.


\textbf{Figure 3}:

The efficiency of reading an edgelist stands out, significantly outpacing the speed of reading a graph stored as CSR, which necessitates an additional conversion step. Surprisingly, the average cost of converting an edgelist to CSR is three times higher than simply reading an edgelist from a file. This underscores the notable speed of modern I/O processes. In the sk-2005 dataset, our approach achieves an impressive reading speed of 1.9 billion edges per second. Notably, it's essential to clarify that the table represents edges as directed, not undirected. To visually represent the edge reading performance across different graphs, we propose the inclusion of a plot illustrating this metric. The sluggishness in CSR reading is justified by its nature as a shared data structure, leading to high contention when multiple threads concurrently attempt to add an edge to the same vertex. The process of adding to CSR structures with small degrees is hampered by false sharing, introducing cache coherency issues that contribute to decreased performance.

We observe that:
- reading edgelist is significantly faster than reading a graph as CSR (this requires and additional conversion step)
- in fact, on average, converting an edgelist to CSR is 3 times more costly than reading and edgelist from file (which is quite surprising indded). This shows that modern IO is fastr (dhooom).
- We are able to read 1.9B edges per second (the table put edges as directed not undirected) in sk-2005.
- Can we have a plot on edge reading performance per graph
- (Justify why CSR reading is slow) converting to CSR is slow becuase it is a shared data structure resulting in high contention beteen thread, when multiple threads attempt to add an edge to the same vertex.
- Also adding to CSR with small degrees suffer from false sharing (cache coherency issue).

\textbf{Figure 4}:



Our observations reveal:

Reading edgelists exhibits impressive scalability, achieving a remarkable 25x scale on 32 threads.
However, at 64 threads, performance is impacted by NUMA effects.
Notably, reading CSR does not scale as well, primarily due to issues such as false sharing and contention.
The influence of NUMA effects is also discernible in CSR reading performance.
The scaling pattern follows multiples of 2 (linear Y-axis, logarithmic X-axis).

% There is a paper called "Efficient Memory Mapped File I/O for In-Memory File Systems" on the topic - where Choi et al. working at Samsung say that mmap() is a good interface for fast IO (in contrast to file streams) and propose async map-ahead based madvise() for low-latency NVM storage devices. The also have a good slides on this - where their (modified) extended madvice obtains ~2.5x better performance than default mmap() by minimizing the number of page faults.
% I tried parsing integers from text and saving into per-thread integer lists. To over-allocate memory for per-thread integer-lists i use sufficient size mmap() instead of malloc(). Even if i over-allocate, due to virtual memory, only memory as much i need is actually used.


\section{Conclusion}
\label{sec:conclusion}
In conclusion, this study introduces GVEL, an optimized approach for efficient Edgelist reading and Compressed Sparse Row (CSR) conversion. Utilizing memory-mapped IO, a technique known for its efficiency in handling large datasets, GVEL outperforms established frameworks such as Hornet, Gunrock, and PIGO by $78\times$, $112\times$, and $1.8\times$ respectively, in CSR reading. For Edgelist reading, GVEL demonstrates a $2.6\times$ speedup over PIGO, and achieves a Edgelist read rate of $1.9$ billion edges/s. Looking ahead, future work could focus on extending GVEL to accommodate dynamic graphs which evolve over time, instead of CSR which is not amenable to dynamic updates. This would contribute to enhancing the adaptability and versatility of GVEL in dynamic graph processing scenarios.


%% The acknowledgments section.
\begin{acks}
I would like to thank Prof. Kishore Kothapalli and Balavarun Pedapudi for their support.
\end{acks}

%% Bibliography style to be used, and the bibliography file.
\bibliographystyle{ACM-Reference-Format}
\bibliography{main}
\end{document}
\endinput
%% End of file.




%% NOTES:
%% - Parallelization seems to be not efficient for small batch updates.
%% - Discuss about conflicting updates
%% - 


%% TODO:
%% - Scale up the size of the graphs
%% - Move experiments to a better server
%% - Include a weak- and strong- scalabiilty plot: run the expt from 2 to 128 threads
%% - overall space planning
%% - add a few lines on novelty of the paper.
%% - table comparison of related work
%% - Include a section on preliminaries that talks about the various algorithmic ideas (Louvain, Label Propagation)

%% Workplan:
%% - KK -- Read Introduction, Related Work,
%% - Dip Sankar -- Approach -- summarize the main algorithmic ideas,
%% - Subhajit -- Results -- Plots, scalability, Dataset, experiments, implementation details,
