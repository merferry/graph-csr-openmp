\subsection{Graph Storage and In-Memory Formats}

Graphs can be stored in either text or binary formats. Text formats, such as Edgelists, provide a simple representation where each line denotes an edge between two vertices (optionally including edge weights and other attributes). While text formats are space-inefficient compared to binary formats, their readability and ease of sharing contribute to their widespread adoption in the public domain. Matrix Market (MTX) is a standardized text format with headers specifying the matrix's properties, making it suitable for various sparse matrix representations.

In-memory graph formats, on the other hand, optimize for efficient data access during computation. These include Edgelists, which mirror the structure of the Edgelist storage format. To improve data locality, the source vertex, target vertex, and edge weights may be stored in separate arrays. This format is easily convertible from Edgelist storage, and is suitable for graph algorithms requiring edge-oriented computation. Compressed Sparse Row (CSR) is another popular in-memory format that is optimal for vertex-oriented algorithms, such as traversal. It stores graph data in three arrays: one for offsets to the list of target vertices / edge weights of each vertex, one for target vertices (for each vertex), and one for edge weights (for each vertex).




\subsection{Memory-Mapped I/O}

Memory-mapped I/O (MMIO) is an access method that maps files or file-like resources to a memory region, providing applications with data access through memory semantics. \texttt{mmap()}, a key system call in memory-mapped I/O, is used to map files into memory, establishing a virtual memory mapping between the process's address space and the file or device. When a thread accesses a page that has not yet been loaded from the file, a page fault occurs, and the thread is put into sleep by the kernel until data from the file has been loaded into the page. The \texttt{madvise()} system call allows programmers to advise the kernel about their expected access patterns for the mapped region, optimizing performance. Subsequently, \texttt{munmap()} is employed to unmap the region, freeing up resources.

MMIO has reduced overhead and eliminates copies between kernel and user space \cite{malliotakis2021hugemap, papagiannis2020optimizing}. Hot data typically resides in main memory, leveraging the benefits of a large cache, while cold data are stored on devices such as HDDs and SSDs \cite{song2016efficient}. The increasing adoption of MMIO is driven by its superior performance, particularly in comparison to file semantics like read/write operations which introduces significant overhead for fast storage \cite{yoshimura2019evfs, enberg2022transcending}. 

\ignore{Memory mapping is a mechanism that maps a file or part of a file into the virtual memory space, so that files on the disk can be accessed as if they were in memory \cite{lin2014mmap}. There are many advantages of using memory mapping, especially when processing large files. - Reading from and writing to a memory-mapped file do not require the data to be copied to and from a user-space buffer while standard read/write do. - Aside from any potential page faults, reading from and writing to a memory-mapped file do not incur any overhead due to context switching. - When multiple processes map the same data into memory, they can access that data simultaneously. Read-only and shared writable mappings are shared in their entirety; private writable mappings may have their not-yet-COW (copy-on-write) pages shared \cite{lin2014mmap}.}




\subsection{Stream Operations in C++}

The Standard Template Library (STL) in C++ offers a suite of stream classes and methods. C++ stream classes include \texttt{ifstream} for input file streams and \texttt{ofstream} for output file streams. The \texttt{istream} and \texttt{ostream} interfaces provide a common set of methods for reading and writing data. The \texttt{>>} and \texttt{<<} operators are overloaded for various data types and provide read and writing of data from and to streams, while \texttt{getline()} method reads a line from the stream. C++ streams use an internal buffer to optimize I/O operations.




\subsection{Number Parsing Methods in C++}

C++ provides several methods for converting textual data to numerical representations. \texttt{sscanf()} is a powerful method for parsing formatted input strings. It operates similarly to `printf` and allows developers to define a format specifier that describes the expected structure of the input string. On the other hand, \texttt{strtoull()} and \texttt{strtod()} offer a simple and direct approach for converting strings to unsigned long long integers and double-precision floating-point numbers, respectively. Performance-wise, \texttt{sscanf()} might involve a bit more overhead due to the need to interpret format specifiers, while \texttt{strtoull} and \texttt{strtod} provide straightforward conversions without the need for format strings.
