\subsection{PageRank}
\label{sec:PageRank}

Consider a directed graph $G(V, E, w)$, with $V$ ($n = |V|$) as the set of vertices and $E$ ($m = |E|$) as the set of edges. The PageRank $R[v]$ of a vertex $v \in V$ in this graph measures its importance based on incoming links and their significance. Equation \ref{eq:pr} defines the PageRank calculation for a vertex $v$ in $G$. $G.in(v)$ and $G.out(v)$ represent incoming and outgoing neighbors of $v$, and $\alpha$ is the damping factor (usually $0.85$). Initially, each vertex has a PageRank of $1/n$, and the \textit{power-iteration} method updates these values iteratively until they converge within a specified tolerance $\tau$, indicating that convergence has been achieved.

% \begin{equation}
% \label{eq:pr}
%     R[v] = \alpha \times \sum_{u \in G.in(v)} \frac{R[u]}{|G.out(u)|} + \frac{1 - \alpha}{n}
% \end{equation}




\subsection{Gini coefficient}

Gini coefficient $G$ is a value which represents income/wealth inequality within a nation or group. It ranges from $0$ to $1$, with $0$ representing total equality and $1$ representing total inequality. It is calculated from the Lorenz curve, which plots cumulative income/wealth against cumulative number of households/people. It is calculated using Equation \ref{eq:gini}, where $A$ is the area between the line of perfect equality and the Lorenz curve, and $B$ is the total area under the line of perfect equality.

% \begin{equation}
% \label{eq:gini}
%   G = \frac{A}{A+B}
% \end{equation}

\begin{table}[!ht]
  \centering
  \caption{In our experiments, we use a list of 17 graphs. Each graph has its edges duplicated in the reverse direction to make them undirected, and a weight of 1 is assigned to each edge. The table lists the total number of vertices ($|V|$), total number of edges ($|E|$) after making the graph undirected, and the file size ($F_{size}$) for each graph. The number of vertices and edges are rounded to the nearest thousand or million, as appropriate.}
  \label{tab:dataset}
  \begin{tabular}{|c||c|c|c|}
    \toprule
    \textbf{Graph} &
    \textbf{\textbf{$|V|$}} &
    \textbf{\textbf{$|E|$}} &
    \textbf{\textbf{$F_{size}$}} \\
    \midrule
    \multicolumn{4}{|c|}{\textbf{Web Graphs (LAW)}} \\ \hline
    indochina-2004$^*$ & 7.41M & 341M & 2.9 GB \\ \hline
    arabic-2005$^*$ & 22.7M & 1.21B & 11 GB \\ \hline
    uk-2005$^*$ & 39.5M & 1.73B & 16 GB \\ \hline
    webbase-2001$^*$ & 118M & 1.89B & 18 GB \\ \hline
    it-2004$^*$ & 41.3M & 2.19B & 19 GB \\ \hline
    sk-2005$^*$ & 50.6M & 3.80B & 33 GB \\ \hline
    \multicolumn{4}{|c|}{\textbf{Social Networks (SNAP)}} \\ \hline
    com-LiveJournal & 4.00M & 69.4M & 480 MB \\ \hline
    com-Orkut & 3.07M & 234M & 1.7 GB \\ \hline
    \multicolumn{4}{|c|}{\textbf{Road Networks (DIMACS10)}} \\ \hline
    asia\_osm & 12.0M & 25.4M & 200 MB \\ \hline
    europe\_osm & 50.9M & 108M & 910 MB \\ \hline
    \multicolumn{4}{|c|}{\textbf{Protein k-mer Graphs (GenBank)}} \\ \hline
    kmer\_A2a & 171M & 361M & 3.2 GB \\ \hline
    kmer\_V1r & 214M & 465M & 4.2 GB \\ \hline
  \bottomrule
  \end{tabular}
  \end{table}

