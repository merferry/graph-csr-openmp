In conclusion, this study introduces GVEL, an optimized approach for efficient Edgelist reading and Compressed Sparse Row (CSR) conversion. Utilizing memory-mapped IO, a technique known for its efficiency in handling large datasets, GVEL outperforms established frameworks such as Hornet, Gunrock, and PIGO by $78\times$, $112\times$, and $1.8\times$ respectively, in CSR reading. For Edgelist reading, GVEL demonstrates a $2.6\times$ speedup over PIGO, and achieves a Edgelist read rate of $1.9$ billion edges/s. Our techniques may also be useful for converting in-memory Edgelists to CSRs on the fly on parallel devices. This is useful since CSR is an space and locality efficient data structure, while Edgelists are easy to update. Further, we observe that our novel multi-stage CSR construction, as employed in GVEL, is particularly effective on high-degree graphs, due to its ability of minimizing thread contention. However, on graphs with lower average vertex degrees, using the standard single-stage method of CSR construction, such as that used by PIGO, is the suitable approach.

\ignore{Looking ahead, future work could focus on extending GVEL to accommodate dynamic graphs which evolve over time, instead of CSR which is not amenable to dynamic updates. This would contribute to enhancing the adaptability and versatility of GVEL in dynamic graph processing scenarios.}
